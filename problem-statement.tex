\documentclass[letterpaper,10pt,titlepage]{article}

\usepackage{graphicx}                                        
\usepackage{amssymb}                                         
\usepackage{amsmath}                                         
\usepackage{amsthm}                                          
\usepackage{cite}
\usepackage{alltt}                                           
\usepackage{float}
\usepackage{color}
\usepackage{url}
\usepackage{titling}

\usepackage{balance}
\usepackage[TABBOTCAP, tight]{subfigure}
\usepackage{enumitem}
\usepackage{pstricks, pst-node}
\usepackage{listings}
\usepackage{color}
\usepackage{tabularx}

\usepackage{geometry}
\geometry{textheight=8.5in, textwidth=6in}

%random comment

\newcommand{\cred}[1]{{\color{red}#1}}
\newcommand{\cblue}[1]{{\color{blue}#1}}

\usepackage{hyperref}
\usepackage{geometry}

\def\name{Krisna Iranwan, Jiongcheng (Roger) Luo and Drew Hamm}

%% The following metadata will show up in the PDF properties
\hypersetup{
  colorlinks = true,
  urlcolor = black,
  pdfauthor = {\name},
  pdfkeywords = {capstone design, problem statement},
  pdftitle = {Problem Statement},
  pdfsubject = {Problem Statement},
  pdfpagemode = UseNone
}
\lstset{language=csh}
\lstset{
	numbers=right, 
	numberstyle=\tiny, 
	breaklines=true,
	numbersep=5pt,
	xleftmargin=0in,
	xrightmargin=.25in,
	tabsize=1,
	frame=single
} 






\begin{document}
\begin{titlepage}
	\centering
	{\scshape\LARGE Oregon State University \par}
	\vspace{2cm}
	{\huge\bfseries Head Worn Display Auto-alignment System\par}
	\vspace{2cm}
	{\Large\itshape \name\par}
	\vfill
	supervised by\par
	Rockwell Collins, Inc

	\vfill

% Bottom of the page
	{\large \today\par}
\end{titlepage}


\begin{abstract}
A Head-up Display (HUD) is a powerful tool that displays real-time flight data to the pilots during the flight environment. The display system allows the pilot to know about the aircraft’s position without looking at other instruments in the cockpit. It indicates the aircraft’s conformal attitude and flight path along with other primary flight data such as airspeed and altitude. Currently, the alignment process for HUDs requires specialized equipment and epoxy which is time consuming, costly, and interrupts production line progress for the original equipment manufacturer (OEM). The resulting HUD alignment, while precise, does not compensate for airframe droop during flight. Rockwell Collins looks forward to a new alignment methodology utilizing an inexpensive microelectromechanical systems (MEMS) inertial reference unit (IRU) mounted onto the HUD to infer alignment using data from the aircraft’s precisely mounted and aligned IRU. This project works on a solution that utilizes the data from both the inexpensive MEMS IRU and the aircraft mounted IRU to develop a near real-time algorithm, which aims to minimize the HUD error and generate higher accuracy symbology.  

\end{abstract}

\section*{Problem Definition}
Accurate data is critical in aviation field. Pilots nowadays relies heavily on the aviation system to ensure their flight safety. When pilots’ life are on the line, it is really important to ensure the accuracy of the aviation data. Rockwell Collins has done an amazing job in ensuring the accuracy of their aviation system. The current system uses the data from the aircraft’s precisely mounted and aligned IRU to update the HUD. However, the current system is unable to detect airframe droop. This airframe droop occurs during flight and shifts the HUD from its precisely aligned position. Rockwell Collins sees a need for aviation HUD to be dynamically realigned during flight. Given the nature of the avionic systems, a solution is a must. More accurate in flight data can be obtained by adding additional aircraft IRUs in separate physical locations within the aviation system. However, a drawback to this approach is the high cost of a single IRU. As a result, Rockwell Collins plans to explore the potential of mounting an inexpensive MEMS IRU onto the HUD itself and comparing this new output with the output of the aircraft’s precisely mounted and aligned IRU in order to quantify the HUDs alignment error. Our goal is to develop a near real-time algorithm that will be able to determine the correct alignment dynamically during a flight environment by using data from both the inexpensive MEMS IRU and the aircraft mounted IRU.

\section*{Problem Solution}
The desired solution is an algorithm that attempts to dynamically calculate the correct HUD alignment in realtime using the data from HUD mounted MEMS IRU and aircraft IRU. Rockwell Collins desires a dynamic solution for alignment which utilizes inertial sensors to constantly measure and minimize the HUD alignment error with respect to the aircraft boresight. Specifically, the algorithm will compare the quaternion outputs from a precise and pre-aligned aircraft aircraft IRU with an output from inexpensive MEMS IRU to quantify the alignment error with respect to the aircraft boresight. From the comparison, we can quantify and find correlation of the alignment error in almost real time. This algorithm will compensate the airframe droop during flight, which result in more dynamic alignment system. The desired algorithm attempt to solve airframe droop problem on the fly while meeting the desired accuracy level of one milliradian. The end result of this project is to create a demonstration system to show a higher accuracy symbology output from the algorithm when coupled with two representative sensors (possibly simulated) and confirm against a known reference.

\section*{Problem Matrices}
This software program will be able to measure and calculate precise alignment data for an aircraft’s conformal attitude and flight path along with other primary flight data such as airspeed and altitude after a given amount of motion. Primarily, a functional data transfer is fundamental for the entire process, this program should be able to recognize and read two groups of quaternion outputs from both inexpensive MEMS IRU and the aircraft’s IRUs. This can be tested by looking at the data transmission behavior of the program. A successful program should be able to receive the quaternion outputs from both IRUs in almost real time. Next, the algorithm of this program should be able to quantify and find the correlation of the alignment error. The algorithm should be able to generate the aligning data by comparing the quaternion outputs from a precise and pre-aligned aircraft mounted IRU with a quaternion outputs from inexpensive MEMS IRU mounted to the HUD.  The aligning data from this algorithm should correctly compensate the alignment error, within one milliradian of accuracy, caused by airframe drop with respect to the aircraft boresight. An ideal algorithm should run within an accepted range of time and complexity. We may use unit tests for each part of the algorithm to check whether the data are processed correctly, and we may also transform the data into statistical data for deep analysis.


%\newpage
%\nocite{*}
%\bibliography{IEEEabrv,References}
%\bibliographystyle{IEEEtran}


\newpage
	\noindent\begin{tabular}{ll}
	\\[2cm]
	\makebox[2.5in]{\hrulefill} & \makebox[2.5in]{\hrulefill}\\
	Client Signature& Date\\[8ex]% adds space between the two sets of signatures
	\makebox[2.5in]{\hrulefill}\\
	Developer\\[8ex]
	\makebox[2.5in]{\hrulefill}\\
	Developer\\[8ex]
	\makebox[2.5in]{\hrulefill}\\
	Developer\\[8ex]
	\end{tabular}

\end{document}