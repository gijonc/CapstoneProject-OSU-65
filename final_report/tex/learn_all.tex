\newpage
\section{What did we learn from all this?}
	\subsection{Jiongcheng Luo}
		\begin{itemize}
			\item \textbf{What technical information did you learn?}
			\\ Throughout this course, a new technology I learned is Latex formatting. Throughout the project, I learned a lot about manufacturing an aircraft, especially its Head-up display system and Inertial Measurement Unit. I learned how an IMUs works and how we could apply it into a useful product. Specifically, from the perspective of software, I learned how to program Arduino and using VPython library.\\

			\item \textbf{What non-technical information did you learn?}
			\\I learned how to collaborate with a team from everyone did not know each other to everyone made a project together, specifically are the skills of communication, cooperation with co-workers. In addition, we had change to contact with a real company client, I learned a lot how a real industry deal with new technologies.  Throughout working on the project, I learned how to compare and choose the tools or technologies to be used for our project.\\

			\item \textbf{What have you learned about project work?}
			\\I learned that before doing a project, a team should have well discussion and make sure everyone understands what this project is doing. After that, a detailed, comprehensive plan needs to be made, during design procedures, all documents need to keep track of any changes and new ideas. During software development, using agile methodology is preferred, so that every part of the software can be modified easily.\\

			\item \textbf{What have you learned about project management?}
			\\A project first should be designed comprehensively including all details, graphs, diagrams, examples, documents etc. And during the implementation, it’s necessary to use version control for all software programs especially for a program that with complicated structure. Everyone can work on a different version for implementing different purpose or portion of the project, a completed version should a merge of all completed parts by different version and the merge needs to be agreed by all team members.\\

			\item \textbf{What have you learned about working in teams?}
			\\I learned that everyone in a team must has unique background knowledge, interests and everyone has different focus field, therefore, in order to get our project done successfully, we need to know each other well, including the lack and expertise of everyone, so that everyone can work on what they good at and everyone can complement for each other.\\

			\item \textbf{If you could do it all over, what would you do differently?}
			\\First, I would get started working on the project earlier, we had spent too much time on the design procedures and researching, then we found that we did not have enough time to accomplish the plans since more problems came up with we moving forward. Second, I would use version control system to keep track of our program, since we had met a problem that we did not save the previous version and we overwrote it, and as a result, we lose the version that was working before, so that we had to redo it all over again. Third, I would meet with our clients more often in regards to technical discussion, we had a lot of issues and problems that we did not know how to solve at all, and we had spent too much time in researching and attempting, that leads to that we could not complete some parts of the project at the end. Fourth, I would try using more IMUs for testing purpose, even though our result showed that working, it did not meet our initial expectation, more IMUs would more accurate.\\
		\end{itemize}

	\subsection{Drew Hamm}
		\begin{itemize}
			\item \textbf{What technical information did you learn?} \\
				This project ended up touching on quite a few topics that were previously unfamiliar to me.
				First, I had to learn about Tait-Bryan angles and how the different conventions worked such as for representing and aircrafts heading, elevation, and bank.
				Next, I had to learn about quaternions and several different filter algorithms, Madgwick and Mahony, for converting our 9-axis data into a usable form.
				Since our work focused primarily hardware, I had to learn how to configure and communicate with our devices; which required extensive reading into register mapping and other documentation.
				Additionally, I learned how to work with accelerometers, gyroscopes, and magnetometers.
				Lastly, I learned improved technical writing skills along with Latex formatting. \\

			\item \textbf{What non-technical information did you learn?} \\
				Primarily, throughout this project I had to learn how to work effectively with a new team.
				Communication amongst the team as well as with our client played heavily into our means to achieve a successful outcome.
				In order to complete tasks on time, especially after falling behind initially, improving my time management skills was crucial.
				Lastly, the skills required for expo such as communicating ideas from technical topics to diverse audiences. \\

			\item \textbf{What have you learned about project work?} \\
				Although I already knew of the importance that communication had in project work, I am now placing even greater emphasis on it for my later project work.
				I learned of the importance of creating tasks that can be developed independently.
				Creating tasks that can be effectively delegated amongst team members to play to everyone's strengths is something I would like to strive for in the future.
				Lastly, I learned how important a frequently updated timeline that includes milestones and well thought out tasks can be. \\

			\item \textbf{What have you learned about project management?} \\
				As for project management, I learned that it's important to keep a direct line of communication open between management and the team.
				Similarly to having a direct line of communication, it's important to maintain an updated timeline of the team's progression. \\

			\item \textbf{What have you learned about working in teams?} \\
				Since we made a poor first impression with our client, our team had to come together to overcome that initial problem.
				I learned how effective a united front could be to resolving issues. \\

			\item \textbf{If you could do it all over, what would you do differently?} \\
				First, I would like to setup version control early.
				Considering our project, there was some ambiguity as to whether we should develop in the open or not.
				Second, I would try to actively apply the skills I learned in my software development classes to promote collaboration and a sense of direction while building momentum as milestones were reached.
				Next, it would have been helpful to send more of my questions towards our client.
				By asking questions often, we could have improved our utilization of their technical experience to better guide my efforts.
				Unfortunately, we did not test our project soon enough and ended up going down a wrong path.
				If I had to do it all over, I would have tested early and often and considered following test driven development.
				Another change that would have greatly improved our results would have been to create an improved timeline that was updated frequently.
				An improved timeline would consist of tasks that could be developed independently.
				As with independent tasks, the timeline should consist of whom is responsible for which task; with proper thought put into each others strengths.
				Lastly, I would have tried to get access to not only the actual IRU that is used within an aircraft, but also the HUD that we were supposed to design our proof of concept solution for. \\
		\end{itemize}

	\subsection{Krisna Irawan}



