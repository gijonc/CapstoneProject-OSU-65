\newpage
\section{What did we learn from all this?}
	\subsection{Jiongcheng Luo}
		\begin{itemize}
			\item \textbf{What technical information did you learn?}
			\\ Throughout this course, a new technology I learned is Latex formatting. Throughout the project, I learned a lot about manufacturing an aircraft, especially its Head-up display system and Inertial Measurement Unit. I learned how an IMUs works and how we could apply it into a useful product. Specifically, from the perspective of software, I learned how to program Arduino and using VPython library.\\

			\item \textbf{What non-technical information did you learn?}
			\\I learned how to collaborate with a team from everyone did not know each other to everyone made a project together, specifically are the skills of communication, cooperation with co-workers. In addition, we had change to contact with a real company client, I learned a lot how a real industry deal with new technologies.  Throughout working on the project, I learned how to compare and choose the tools or technologies to be used for our project.\\

			\item \textbf{What have you learned about project work?}
			\\I learned that before doing a project, a team should have well discussion and make sure everyone understands what this project is doing. After that, a detailed, comprehensive plan needs to be made, during design procedures, all documents need to keep track of any changes and new ideas. During software development, using agile methodology is preferred, so that every part of the software can be modified easily.\\

			\item \textbf{What have you learned about project management?}
			\\A project first should be designed comprehensively including all details, graphs, diagrams, examples, documents etc. And during the implementation, it’s necessary to use version control for all software programs especially for a program that with complicated structure. Everyone can work on a different version for implementing different purpose or portion of the project, a completed version should a merge of all completed parts by different version and the merge needs to be agreed by all team members.\\

			\item \textbf{What have you learned about working in teams?}
			\\I learned that everyone in a team must has unique background knowledge, interests and everyone has different focus field, therefore, in order to get our project done successfully, we need to know each other well, including the lack and expertise of everyone, so that everyone can work on what they good at and everyone can complement for each other.\\

			\item \textbf{If you could do it all over, what would you do differently?}
			\\First, I would get started working on the project earlier, we had spent too much time on the design procedures and researching, then we found that we did not have enough time to accomplish the plans since more problems came up with we moving forward. Second, I would use version control system to keep track of our program, since we had met a problem that we did not save the previous version and we overwrote it, and as a result, we lose the version that was working before, so that we had to redo it all over again. Third, I would meet with our clients more often in regards to technical discussion, we had a lot of issues and problems that we did not know how to solve at all, and we had spent too much time in researching and attempting, that leads to that we could not complete some parts of the project at the end. Fourth, I would try using more IMUs for testing purpose, even though our result showed that working, it did not meet our initial expectation, more IMUs would more accurate.\\
		\end{itemize}

	\subsection{Drew Hamm}
		\begin{itemize}
			\item \textbf{What technical information did you learn?} \\
				This project ended up touching on quite a few topics that were previously unfamiliar to me.
				First, I had to learn about Tait-Bryan angles and how the different conventions worked such as for representing and aircrafts heading, elevation, and bank.
				Next, I had to learn about quaternions and several different filter algorithms, Madgwick and Mahony, for converting our 9-axis data into a usable form.
				Since our work focused primarily hardware, I had to learn how to configure and communicate with our devices; which required extensive reading into register mapping and other documentation.
				Additionally, I learned how to work with accelerometers, gyroscopes, and magnetometers.
				Lastly, I learned improved technical writing skills along with Latex formatting. \\

			\item \textbf{What non-technical information did you learn?} \\
				Primarily, throughout this project I had to learn how to work effectively with a new team.
				Communication amongst the team as well as with our client played heavily into our means to achieve a successful outcome.
				In order to complete tasks on time, especially after falling behind initially, improving my time management skills was crucial.
				Lastly, the skills required for expo such as communicating ideas from technical topics to diverse audiences. \\

			\item \textbf{What have you learned about project work?} \\
				Although I already knew of the importance that communication had in project work, I am now placing even greater emphasis on it for my later project work.
				I learned of the importance of creating tasks that can be developed independently.
				Creating tasks that can be effectively delegated amongst team members to play to everyone's strengths is something I would like to strive for in the future.
				Lastly, I learned how important a frequently updated timeline that includes milestones and well thought out tasks can be. \\

			\item \textbf{What have you learned about project management?} \\
				As for project management, I learned that it's important to keep a direct line of communication open between management and the team.
				Similarly to having a direct line of communication, it's important to maintain an updated timeline of the team's progression. \\

			\item \textbf{What have you learned about working in teams?} \\
				Since we made a poor first impression with our client, our team had to come together to overcome that initial problem.
				I learned how effective a united front could be to resolving issues. \\

			\item \textbf{If you could do it all over, what would you do differently?} \\
				First, I would like to setup version control early.
				Considering our project, there was some ambiguity as to whether we should develop in the open or not.
				Second, I would try to actively apply the skills I learned in my software development classes to promote collaboration and a sense of direction while building momentum as milestones were reached.
				Next, it would have been helpful to send more of my questions towards our client.
				By asking questions often, we could have improved our utilization of their technical experience to better guide my efforts.
				Unfortunately, we did not test our project soon enough and ended up going down a wrong path.
				If I had to do it all over, I would have tested early and often and considered following test driven development.
				Another change that would have greatly improved our results would have been to create an improved timeline that was updated frequently.
				An improved timeline would consist of tasks that could be developed independently.
				As with independent tasks, the timeline should consist of whom is responsible for which task; with proper thought put into each others strengths.
				Lastly, I would have tried to get access to not only the actual IRU that is used within an aircraft, but also the HUD that we were supposed to design our proof of concept solution for. \\
		\end{itemize}

	\subsection{Krisna Irawan}
	\begin{itemize}
			\item \textbf{What technical information did you learn?} \\
				I learned how MPU works. Before I took this course, I did not even know that this MPU is consist of gyroscope, accelerometer, and magnetometer. Taking this course was a huge learning curve for me. I also learned about quaternion, Proper Euler angles, and Tait–Bryan angles. I never learned about these concepts on any of my OSU classes and I learned it while working with this project. I also learn on how to use Vpython to create the graphical user interface. This language was straightforward and easier than OpenGL. I am glad that I learned it in this project. Lastly, I also learned to use LaTex. It was a steep learning curve, but it gets easier when you are used to it. \\

			\item \textbf{What non-technical information did you learn?} \\
				I learned to communicate effectively with my teammates. I learned to respect people time, especially our clients time. Communication is the most important thing during a huge group project like this. By working with a real industry client, Rockwell Collins, I also get a taste on how a real industry works. The documents that I wrote for this class taught me that project manager is not an easy job. Project manager has to deal with a lot of administration and documentation of the project, while at the same time, oversees the project. \\

			\item \textbf{What have you learned about project work?} \\
				I learned that planning is crucial for the success of the project. It was important for the team to be on the same page, and have the same goal. It was also important to consider that there is a risk in every step of the project. Thus, starting early and having an extra time on the schedule to work on the error was crucial. I also learned that documents that we wrote should reflect the changes on our project at all time, thus reducing any miscommunication between teammates.  \\

			\item \textbf{What have you learned about project management?} \\
				Documentations took a lot work and effort to do. However, documentations were also crucial for the success of the project. These documents were our main communication with the clients. Our clients gave valuable input and feedback based on these documents. Hence, it was important to keep it up to date. Version control was also important for the success of the team. A lot of time for debugging and merging can be reduced if we have a good version control. Also, by having a good version control, we can confidently work on a new concept without having a fear of losing or messing-up the previous work, and when we do messed-up, we can always go back to the old version. \\

			\item \textbf{What have you learned about working in teams?} \\
				Everybody in a team have their own weaknesses and strengths. To work well together as a team, it was important to know the strengths and weaknesses of each team member from early on. It was important for the team to have each other backs and complement each other. It was also important to communicate effectively and pro-actively. Effective communication was the key for a successful team project. \\

			\item \textbf{If you could do it all over, what would you do differently?} \\
				I would like to start early on the project. There were a lot of things that our team should learn for this project, which cuts down the implementation time of this project. I think by starting early, we can get some extra time to work on the implementation of this project. I also want to have a risk management plan while planning on the timeline for this project. It is important to consider that there is a risk in every step of the project. Thus, starting early and having an extra time on the schedule to work on the error was crucial. There were a lot of time where we created a tight schedule without even considering what would go wrong with that step of the project. When that step goes wrong, out team scramble all over the place and panic. This made us cut the next step implementation time to work on debugging the current step. This method was proven to be inefficient and stressful. Thus, if I could do it all over again, I would like to have a better risk management plan. \\
				
				Next, I also want to have a better version control. There was a time when we have a working version and start to work on the next implementation step. The next thing we know, the next implementation step did not work and we have no way to go back to the old version. It was frustrating for our team, trying to make the code work like it used to be. This experience makes me realize that version control is important. \\
				
				Finally, I want to have a better communication with the clients. During the last two terms, our meeting with the clients was minimal. I think this is caused by the lack of communication from our team to the clients. I think it is important to keep in touch with our clients on a weekly basis. A small email updates should be good to serve this purpose. In this email, our team can also ask the clients questions that we might have for the project. There was a lot of time where our team spend too much time on a problem without having any result. I think our clients’ technical expertise can guide us through this problem, and create more time for us to further refine the project. \\
		\end{itemize}



