\section{Definitions}
\begin{itemize}
	\item \textbf{HUD}
 	\begin{adjustwidth}{2.5em}{0pt}
 	A Head-Up Display (HUD) is a transparent display placed in front of a pilot’s head position in the cockpit of the aircraft. A HUD presents critical flight information to the pilots during the flight environment by using graphical, numerical and symbolical data \cite{hud}.
 	\\
 	\end{adjustwidth}

	\item \textbf{Airframe Droop}
 	\begin{adjustwidth}{2.5em}{0pt}
	As the center of the aircraft is pushed up due to lift, the nose of the aircraft is pulled down by its weight. The result of these conflicting forces causes a bend in the aircraft \textquotesingle s frame.
	\\
 	\end{adjustwidth}

	\item \textbf{Conformal Attitude}
 	\begin{adjustwidth}{2.5em}{0pt}
	A method of presenting flight information on the HUD. With conformal attitude presentation, the displayed information on the HUD is presented with respect to the real world (outside scene in front of the aircraft), and the presentation is dependent on the head position of the pilot \cite{conformal_atti}.
	\\
 	\end{adjustwidth}

 	\item \textbf{Real-time}
 	\begin{adjustwidth}{2.5em}{0pt}
	In this system, the algorithm should recognize and precisely align the alignment error for IMU output within minimal time (e.g., 500 milliseconds).\\
 	\end{adjustwidth}

 	\item \textbf{IRU/IMU}
 	\begin{adjustwidth}{2.5em}{0pt}
	An Inertial Reference Unit (IRU), sometimes it’s called an Inertial Measurement Unit (IMU) is a device consists of inertial sensors typically including MEMS gyroscopes, accelerometers and compasses. These MEMS sensors measure and provide data of an aircraft \textquotesingle s velocity, acceleration and orientation \cite{iru}.
	\\
 	\end{adjustwidth}

 	\item \textbf{MEMS}
 	\begin{adjustwidth}{2.5em}{0pt}
	An Inertial Reference Unit (IRU), sometimes it’s called an Inertial Measurement Unit (IMU) is a device consists of inertial sensors typically including MEMS gyroscopes, accelerometers and compasses. These MEMS sensors measure and provide data of an aircraft \textquotesingle s velocity, acceleration and orientation \cite{mems}.
	\\
 	\end{adjustwidth}

 	\item \textbf{Accelerometer}
 	\begin{adjustwidth}{2.5em}{0pt}
	An electromechanical device that measures the amount of static acceleration due to gravity, in other word, the rate of change in velocity of the mounted object \cite{accelerometer}.
	\\
 	\end{adjustwidth}

 	\item \textbf{MEMS Gyroscope}
 	\begin{adjustwidth}{2.5em}{0pt}
	An electromechanical device that measures rotational motion/angular velocity, in other word, the speed of rotation of the mounted object \cite{gyroscope}.
	\\
 	\end{adjustwidth}

 	\item \textbf{I2C}
 	\begin{adjustwidth}{2.5em}{0pt}
	The Inter-integrated Circuit (I2C) Protocol is a protocol intended to allow multiple “slave” digital integrated circuits (“chips”) to communicate with one or more “master” chips \cite{i2c}.
	\\
 	\end{adjustwidth}

 	\item \textbf{SPI}
 	\begin{adjustwidth}{2.5em}{0pt}
	Serial Peripheral Interface (SPI) is a synchronous serial data protocol used by microcontrollers for communicating with one or more peripheral devices quickly over short distances. It can also be used for communication between two microcontrollers \cite{spi}.
	\\
 	\end{adjustwidth}

 	\item \textbf{UART}
 	\begin{adjustwidth}{2.5em}{0pt}
	A universal asynchronous receiver/transmitter (UART) is a block of circuitry responsible for implementing serial communication. On one end of the UART is a bus of eight-or-so data lines (plus some control pins), on the other is the two serial wires - RX and TX \cite{uart}.
	\\
 	\end{adjustwidth}

 	\item \textbf{Multiplexer}
 	\begin{adjustwidth}{2.5em}{0pt}
	A MUX (Multiplexer) is a combinational logic circuit designed to switch one of several input lines through to a single common output line by the application of a control signal \cite{mux}.
	\\
 	\end{adjustwidth}

 	\item \textbf{DMP}
 	\begin{adjustwidth}{2.5em}{0pt}
	A Digital Motion Processor (DMP is a programmable chip within the SparkFun MPU-9250 IMU MEMS sensor. Allows for filtering output as well as converting to quaternion output without the need of an additional microcontroller \cite{dmp}.
	\\
 	\end{adjustwidth}

 	\item \textbf{Alignment Algorithm}
 	\begin{adjustwidth}{2.5em}{0pt}
	The uses of mathematical and logical equations to align the input data from the IRUs based on the data from previous precisely pre-aligned IRUs.
	\\
 	\end{adjustwidth}

 	\item \textbf{Symbology}
 	\begin{adjustwidth}{2.5em}{0pt}
	The use and interpretation of symbols or special characters, in order for representing the corresponding flight data.
	\\
 	\end{adjustwidth}

 	\item \textbf{Milliradian}
 	\begin{adjustwidth}{2.5em}{0pt}
	1 Milliradian (MRAD) = 0.001 radian, approximately 0.057296 degrees.
	\\
 	\end{adjustwidth}

 	\item \textbf{Quaternion}
 	\begin{adjustwidth}{2.5em}{0pt}
	Four-element vector that is used to encode any rotation in a 3D coordinate system.
	\\
 	\end{adjustwidth}

 	\item \textbf{Altitude}
 	\begin{adjustwidth}{2.5em}{0pt}
	A distance measurement (usually in vertical) between ground and the aircraft.
	\\
 	\end{adjustwidth}

 	\item \textbf{Latitude}
 	\begin{adjustwidth}{2.5em}{0pt}
	AAn angular distance shown as a horizontal line, in degrees, minutes, and seconds of a point north or south of the Equator. Lines of latitude are often referred to as parallels \cite{latAndLong}.
	\\
 	\end{adjustwidth}

 	\item \textbf{Longitude}
 	\begin{adjustwidth}{2.5em}{0pt}
	An angular distance shown as a vertical line, in degrees, minutes, and seconds, of a point east or west of the Prime (Greenwich) Meridian. Lines of longitude are often referred to as meridians \cite{latAndLong}.
 	\end{adjustwidth}
\end{itemize}