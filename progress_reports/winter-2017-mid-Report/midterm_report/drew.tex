\newpage 
\section{Drew Hamm}
\subsection{Purpose}
We have made progress towards our demonstration system of aligning data to a HUD in realtime.
The purpose of this section will be to describe our current stage of progress, the remaining work that needs to be completed as well as problems we have come across and the solutions that were implemented.\\

\subsection{Current Stage}
Our primary focus at this point in our project has been on the hardware required to build the demonstration system, communication between devices and data manipulation.
Due to a logistical error we were late to start digging into the meat of our project, however we have made significant progress in these last few weeks.
The description of the current stage of our project will follow the timeline in which progress was made.
First, a description of acquiring and modifying hardware at the physical level.
Next, the configuration required for communication between devices.
Lastly, data manipulation in terms of improvements, conversions and use.\\

We have successfully acquired three Metro Mini microcontrollers to meet the one Metro Mini requirement.
The additional microcontrollers allow us to speed up development as each group member may work independently to solve and test independent tasks.
Additionally, these extra microcontrollers provide tolerance in case of single hardware failure or project redesign.
Second, we have successfully acquired three MPU-9250 IMUs to meet the two IMU requirement.
While these additional IMUs allow for independent development by group members to some extent, our demonstration system requires at least two per system.
Again, we have some leeway in case of hardware failure as well as allowing us to combine additional MEMs to reduce error.\\

Although the hardware acquired are capable of meeting our projects requirements, some modifications were needed whereas some were made for ease of use.
An ease of use modification was carried out by soldering pins to the bottom of both IMUs and microcontroller to reduce development time by allowing rapid prototyping via breadboard.
A single similar modification was needed on each IMU and the microcontroller.
To allow the IMU's I2C address to be changed dynamically from 0x68 to 0x69 via its AD0 pin, a jumper on the bottom of the MEM had to be resoldered.
Initially, the IMU's I2C address was preset to 0x68 as the jumper connected the AD0 pin to ground.
After the modification was made, the IMU could have its address set to 0x69 by supplying 3.3v to the AD0 pin.
To convert the microcontroller's digital logic pins to 3.3v down from 5v we had to cut and solder a jumper on the bottom of the microcontroller.\\

After wiring the devices together some configuration was required before communication via I2C could work appropriately.
In order to access magnetometer data from individual IMUs we had to configure the magnetometer  to act as a slave to the MEM while disabling the default pass through mode as defined by the MEM.
Without, this modification the separate magnetometers would be indistinguishable on the I2C bus as they use the same address.
When changing the magnetometer to act as a slave to the IMU, the accelerometer and gyroscope were also set this way.
Although all sensors were configured in much the same way, the magnetometer may only be set to return one byte at a time and thus required a slightly different configuration.\\

Now that the communication between devices has been supported, we were able to retrieve and manipulate IMU data.
To improve the accuracy of data being generated by the IMUs, each IMU needed to be initialized and calibrated at bootup.
First, we took a sampling of at-rest readings from both gyroscope and accelerometer after device initialization.
Next, we used these readings to calculate the average offset for each sensor.
Lastly, we stored these offsets in accelerometer and gyro bias registers.\\

With the initial calibration taken care of we were able to output accelerometer, gyroscope and magnetometer data for each MEM via the single I2C bus.
While we are only using two MEMs now, by dynamically changing addresses via each MEMs AD0 pin, we are able to access data from as many MEMs as our microcontroller has digital logic pins running at 3.3v.
To improve accuracy of the data we are able to access, a filtering technique was used.
In our case we chose the Madgwick filter fusion algorithm to create a quaternion-based estimate of device orientation by fusing sensor data.\\

Lastly, now that the IMU's data has been converted to quaternions, we have the ability to use that data for our implementation.
We are able to find the difference between IMUs with a straightforward formula.
To find the difference between IMUs, we take the quaternion data from one IMU and multiplying it by the inverse of the other IMU's quaternion data.
With quaternion manipulation achieved, we are now able to continue moving towards our solution of aligning data.\\

\subsection{Remaining work}
\begin{itemize}
	\item \textbf{Improve offset algorithm}\\
		While we are able to determine the difference between IMUs, we must modify the algorithm to account for both the initial alignment error and the alignment error that would occur during the flight environment.\\
	\item \textbf{Create statistical analysis methods}\\
		By creating statistical analysis methods we will be able to improve the offset algorithm.\\
	\item \textbf{Test individual MEM error}\\
		To ensure we are able to meet the maximum alignment error, we will need to test the IMUs. We may find that additional IMUs will be required to reduce alignment error further.\\
	\item \textbf{Calculate and include sufficient number of MEMS to meet accuracy requirements}\\
		After testing IMU error and comparing it to the described hardware specifications we will use this information and apply formula to determine the minimum number of IMUs that would be required without changing the specific IMU model being used.\\
	\item \textbf{Create demonstration UI showing appropriate symbology and alignment corrections}\\
		Our demonstration system tries to mimic the real world representation by including a US with the appropriate symbology.\\
	\item \textbf{Create physical demonstration system}\\
		Our current progress has left us with an implementation in which both IMUs are on a single breadboard. Although this may be sufficient for initial alignment, dynamic alignment would require the IMUs to move independently from one another.\\
\end{itemize}

\subsection{Problems and Solutions}
\begin{itemize}
	\item \textbf{Problem 1:}
	Using I2C with both IMUs would not work when the IMUs shared the same address.\\

	\textbf{Solution:}
	Apply 3.3v to the IMU's AD0 pin to set its address to 0x69 or ground for 0x68.\\
	
	\item \textbf{Problem 2:}
	To dynamically change the address of each IMU we needed the microcontroller to output 3.3v from its digital logic pins but our microcontroller was set to 5v.\\

	\textbf{Solution:}
	To change the digital logic pins output to 3.3v down from 5v we had to cut a jumper and solder a new connection on the underside of the microcontroller.\\

	\item \textbf{Problem 3:}
	Setting 3.3v to the AD0 pin of an IMU was not changing its address to 0x69.\\

	\textbf{Solution:}
	By soldering a jumper on the IMU we were able to enable the expected AD0 functionality once it's short to ground was removed.\\

	\item \textbf{Problem 4:}
	Setting 3.3v to the AD0 pin of an IMU was not changing its address to 0x69.\\

	\textbf{Solution:}
	By soldering a jumper on the IMU we were able to enable the expected AD0 functionality once it's short to ground was removed.\\

\end{itemize}

\subsection{Experimental Design Description}
We may find that in order to meet the maximum alignment error tolerance without changing the model of IMU being used, we would need to combine additional IMUs.
It is known that by averaging output of multiple IMUs, that precision can be improved by the square root of n, where n is the number of independent measurements.
Optimal precision may be reached with IMUs have equal and opposite biases.
Although we have little control over the individual IMU bias in one direction or another, we will may find it useful to increase the number of IMUs in our demonstration system.