

% ************************ section 1 ************************
\section{Introduction}
\subsection{Purpose}
The primary purpose of the SRS is to create a system using the additional HUD mounted MEMS IRU that will meet installation goals while reducing costs. Costs will be reduced by removing the need to use specialized equipment and epoxy to mount the HUD as well as reducing the overall down time for the original equipment manufacturer. By improving the installation process with this additional sensor, the secondary purpose of the SRS is to use this sensor to detect and correct the alignment error from airframe droop. This project has been initiated by RC to improve their current installation process while aiming to detect and solve for airframe droop during flight.

\subsection{Scope}
The product is called MEMS system prototype. This product will be a near real-time algorithm that will be able to determine the correct alignment dynamically during a flight environment by using data from both the inexpensive MEMS IRU and the aircraft mounted IRU. The product is a proof of concepts that will be a start of a future project for Rockwell Collins. The algorithm is used to determined the value of having additional MEMS IRU in the system and how that affect the accuracy of the HUD data. 

Currently, the HUD obtains data from an aircraft’s mounted device called inertial reference unit (IRU), this IRU outputs precise and aligned data to the HUD. However, the current alignment process requires specialized equipment and epoxy which is time consuming, costly, and interrupts production line progress for the original equipment manufacturer. In addition, the resulting HUD alignment, while precise, does not compensate for airframe droop during flight. Rockwell Collins looks forward to a new alignment methodology utilizing an inexpensive microelectromechanical systems (MEMS) IRU mounted onto the HUD to infer alignment data from the aircraft’s precisely mounted and aligned IRU. This project works on a solution that utilizes the data from both the inexpensive MEMS IRU and the aircraft mounted IRU to develop an algorithm, which aims to output precise and aligned data with reduced installation cost. 

The outcome (aligned-data) of this algorithm should compensate the alignment error correctly, and the alignment error should be within a range of one milliradian. The product 

\subsection{Definitions, Acronyms, and Abbreviations}
...\\\\

\subsection{References}
...\\\\

\subsection{Overview}
...\\\\



% ************************ section 2 ************************

\section{Overall Description}
\subsection{Product Perspective}
...\\\\

\subsection{Product Functions}
...\\\\

\subsection{User Characteristics}
...\\\\

\subsection{Constraints}
...\\\\

\subsection{Assumption and Dependencies}
...\\\\


% ************************ section 3 ************************

\section{Specific Requirements}

