\newpage
	\section{Fall 2016}
%--------------------------------------------------------------------
%							Roger's Blog Post
%--------------------------------------------------------------------
		\subsection{Jiongcheng Luo (Fall 2016)}
		\vspace{0.5cm}

		\begin{center}
			\textbf{Week 3 (Oct. 10 {\textasciitilde{}} 14, 2016)}
		\end{center}
		\begin{itemize}
			\item \textbf{Plans for the coming week}
			\\My primary plan for next week is to complete the problem statement that is satisfied by the clients also have them sign on it. The next step is to understand the problem more in depth in the technical perspective, I hope to illustrate the problem by using mathematical and physics way, and be able to translate into CS problem such as what would be software program that we will build for the project.\\

			\item \textbf{Progress since last week}
			\\The most important progress I've made was I have built a better relationship with my teammates I have known them better. I also have a more clear understanding on the problem that we are trying to solve for the given project. We completed our first draft of problem statement even though it didn't meet the requirement from our clients, but it helped me comprehend the entire problem more deeply.\\

			\item \textbf{Any problems I encountered}
			\\It's hard to follow the agenda of our client, which we always had to wait for an uncertain long time for getting their email back, which may affect to our work progress in the future. In addition, the scheduling within our group is not settled yet, we have not yet set up a fixed time for the group meeting.\\
		\end{itemize}

		\rule{\textwidth}{0.5pt}

		\begin{center}
			\textbf{Week 4 (Oct. 16 {\textasciitilde{}} 21, 2016)}
		\end{center}
		\begin{itemize}
			\item \textbf{Plans for the coming week}
			\\By next week, the primary goal is to get the problem statement fixed as the expectation of the client, that in terms of the consistence of the entire statement. We will include more explanation in plain language to the technical terms. After that, I plan to do more and deeper researches, and start to working on the first design procedures to the problem.\\

			\item \textbf{Progress since last week}
			\\My progress is I understood how a team collaboration is so important to the success of this project, we have been getting closer as an entity and I started know the the character of each team member: what each of them good at and lack, that understanding helps me to know how to make better complement for each of us.\\

			\item \textbf{Any problems I encountered}
			\\The problem I met at this moment is how to improve the relationship between us and the client, since we had an issue that has led to breakdown of our relationship. Besides, I still have unclear problems for our projects such as the I am still not clear about the real time requirement for our algorithm, or like what kind of data we will get for test.\\
		\end{itemize}

		\rule{\textwidth}{0.5pt}

		\begin{center}
			\textbf{Week 5 (Oct. 23 {\textasciitilde{}} 28, 2016)}
		\end{center}
		\begin{itemize}
			\item \textbf{Plans for the coming week}
			\\The primary goal of next week is to complete the requirement document, which requires us to start thinking and planing for working on the project. I also plan to have one or two meetings with our client in regards to the requirement documents.\\

			\item \textbf{Progress since last week}
			\\My biggest progress from last week was that we have completed the problem statement as the expectation from our clients. And by that, I have become more familiar with the terminologies of our project, I have a clear picture in my mind about the problem that we are dealing with.\\

			\item \textbf{Any problems I encountered}
			\\I am still not sure about the resolution to our project, such as what kind of software and hardware we are going to play with. In other word, I am still not clear about our procedures for doing this project.\\
		\end{itemize}

		\rule{\textwidth}{0.5pt}

		\begin{center}
			\textbf{Week 6 (Oct. 31 {\textasciitilde{}} Nov. 5, 2016)}
		\end{center}
		\begin{itemize}
			\item \textbf{Plans for the coming week}
			\\By next week, I hope finish up the requirement document including all the sectiosubsections; and I plan to decided which hardwares (boards) to use for our project and start thinking to purchase. And after decide which board to use, I plan to start looking at more detail of the board and maybe starting do some simple coding simulation.\\

			\item \textbf{Progress since last week}
			\\By writing on the requirement document, I know much better about the specific points of the project such as the detail workflow, input/output of the product and restriction, etc. In addition, we have known about what sort of boards that we are using for building the product, which helped me narrow the learning process so that I know what to look up and learn about.\\

			\item \textbf{Any problems I encountered}
			\\We are still struggling about some of the detail information about the product when writing on the requirement document, many specific points are unknown until we start the implementation.\\
		\end{itemize}

		\rule{\textwidth}{0.5pt}

		\begin{center}
			\textbf{Week 7 (Nov. 7 {\textasciitilde{}} Nov. 11, 2016)}
		\end{center}
		\begin{itemize}
			\item \textbf{Plans for the coming week}
			\\My plan for the next week is to finish up the the technology review, that's not only for the writing part, but also plan to list out and truly understand all the technologies that we may use as well as how to use these technologies for our project. Also, I plan to list out all the hardware that we are going to implement on and prepare to purchase for those.\\

			\item \textbf{Progress since last week}
			\\We have eventually finished the requirement document by last week and move on to the technologies review, by writing up the requirement document, I have a deeper understanding to the restriction and the problem that we may meet during the real implementation process.\\

			\item \textbf{Any problems I encountered}
			\\We still have no solutions or ideas for solving some of the specific problems. For example, we need two groups of input data and one of them represent the correct aligned "Aircraft" IRU data for this project, how do we get the "correct aligned" data, what reference do we take to assume those data we come up is correct?\\
		\end{itemize}

		\rule{\textwidth}{0.5pt}

		\begin{center}
			\textbf{Week 8 (Nov. 14 {\textasciitilde{}} Nov. 18, 2016)}
		\end{center}
		\begin{itemize}
			\item \textbf{Plans for the coming week}
			\\By this week, I plan to start up and hopefully finish a rough draft by the end of the week. I also nail down the all the hardware that we are going to use and send out the list to Kevin. In addition, I plan to send out an email to our client to report our progress and ask about some questions: 1. How to get correct aligned data as reference? 2. Ask for generic HUD symbology picture 3. GitHub Account.\\

			\item \textbf{Progress since last week}
			\\We have finished up our technologies review and we have discuss about the question about how to get correct aligned data as reference, even though we do not if our assumption will be correct or doable or not when doing the real implementation. But we assume we can "make" a group of correct aligned data by manually adjusting it and assume this data is correct, so that we will let the other group of data to be correctly aligned based on this assumption.\\

			\item \textbf{Any problems I encountered}
			\\The problem I had so far is for getting the correct aligned data as reference, and we have to figure out the hardware we going to use.\\
		\end{itemize}


%--------------------------------------------------------------------
%							Drew's Blog Post
%--------------------------------------------------------------------

		\subsection{Drew Hamm (Fall 2016)}
		\vspace{0.5cm}

		\begin{center}
			\textbf{Week 3 (Oct. 10 {\textasciitilde{}} 14, 2016)}
		\end{center}
		\begin{itemize}
			\item \textbf{Plans for the coming week}
			\\ I want to start looking into the hardware we might be working with for this project. Specifically I will be reading the specifications for both the Motion Sensor Evaluation Board: MPU-9250CA-SDK and the Ellipse-D: Miniature Dual GPS INS. Besides familiarizing myself with the hardware I want to learn about Quaternions as advised by our client in order to better understand the output data we will be working with. Lastly, I plan on finishing up my section of the problem statement as well as working with my team to finish it as a whole.\\

			\item \textbf{Progress since last week}
			\\Met up with team members to work on the problem statement and finish a first draft. Received feedback from clients further specifying the project details.\\

			\item \textbf{Any problems I encountered}
			\\I found out that my first understanding of the project was incorrect. At first I thought we were mostly working on a proof of concept. I realized my understanding of hardware error was poor. Since our project requires accurate results, I need to spend some time to understand how much error might be expected from our solution.\\
		\end{itemize}

		\rule{\textwidth}{0.5pt}

		\begin{center}
			\textbf{Week 4 (Oct. 17 {\textasciitilde{}} 21, 2016)}
		\end{center}
		\begin{itemize}
			\item \textbf{Plans for the coming week}
			\\ I'll be working together with the group in order to finish up a new revision of our problem statement. We want to address a couple of our clients concerns in order to get their signed approval. We have a meeting with our client on Wednesday to which I want to prepare questions for. We should also be able to get our problem statement signed so we can move on to working on the requirements document.\\

			\item \textbf{Progress since last week}
			\\Met with group to clear up some miscommunication we had with our client. Worked on on the problem statement along with research on both MEMS IRU and aircraft IRU specifications.\\

			\item \textbf{Any problems I encountered}
			\\The technical writing required to create the problem statement has been difficult. Most of this difficulty is due to having to learn the specialized terminology as well as hardware that we will be working with.\\
		\end{itemize}

		\rule{\textwidth}{0.5pt}

		\begin{center}
			\textbf{Week 5 (Oct. 24 {\textasciitilde{}} 28, 2016)}
		\end{center}
		\begin{itemize}
			\item \textbf{Plans for the coming week}
			\\ I will be working on the requirements document. This will also involve getting together with the group and doing more research in order to fully explain our solution. We will need to get in touch with our client once we have a working draft of the document.\\

			\item \textbf{Progress since last week}
			\\Finished our problem statement and met with our client. Our meeting was productive as it helped to answer a few questions we had as well as to ensure that we are covering everything of importance within our project.\\

			\item \textbf{Any problems I encountered}
			\\Schedule conflicts for myself and the group made this week difficult. Our solution was less meeting in person and more work being done online.\\
		\end{itemize}

		\newpage

		\begin{center}
			\textbf{Week 6 (Oct. 31 {\textasciitilde{}} Nov. 4, 2016)}
		\end{center}
		\begin{itemize}
			\item \textbf{Plans for the coming week}
			\\ Continue working on the requirements document. Hopefully finish by Friday. Send client our finished document.\\

			\item \textbf{Progress since last week}
			\\Met with group members and decided what sections we are each responsible for writing within the requirements document. Started writing the my sections of the document.\\

			\item \textbf{Any problems I encountered}
			\\Some aspects of the project are quite complex and will require extra time to research.\\
		\end{itemize}

		\rule{\textwidth}{0.5pt}

		\begin{center}
			\textbf{Week 7 (Nov. 7 {\textasciitilde{}} Nov. 11, 2016)}
		\end{center}
		\begin{itemize}
			\item \textbf{Plans for the coming week}
			\\ First, I need to decide what to include in the tech review. Next, the group needs to choose who will be responsible for each item. Lastly, start working on the tech review and finish it by Friday.\\

			\item \textbf{Progress since last week}
			\\We finished the requirements document and sent it off to our client.\\

			\item \textbf{Any problems I encountered}
			\\Last week was busy with midterms and other assignments. Both group members and myself struggled to find time to meet and finish the requirements document.\\
		\end{itemize}

		\rule{\textwidth}{0.5pt}

		\begin{center}
			\textbf{Week 8 (Nov. 14 {\textasciitilde{}} Nov. 18, 2016)}
		\end{center}
		\begin{itemize}
			\item \textbf{Plans for the coming week}
			\\ Although we submitted the tech review I want to look into our project for other options we might need to decide on later. Looking for additional items will carry over into starting work on the design document. Planning to meet with group so we can decide what sections everyone will be responsible for.\\

			\item \textbf{Progress since last week}
			\\Finished the tech review. Researched filter techniques for sensor data. Learned about advancements in MEMS quality assurance testing.\\

			\item \textbf{Any problems I encountered}
			\\We originally choose 9 items for the tech review however, one of the items was too straight forward to find alternative solutions. We had a hard time finding an additional item to include so we left the document with only 8. I found that the tech review took more time than expected to complete as the research I had to do was quite complicated.\\
		\end{itemize}

%--------------------------------------------------------------------
%							Krisna's Blog Post
%--------------------------------------------------------------------

		\subsection{Krisna Irawan (Fall 2016)}
		\vspace{0.5cm}

		\begin{center}
			\textbf{Week 3 (Oct. 10 {\textasciitilde{}} 14, 2016)}
		\end{center}
		\begin{itemize}
			\item \textbf{Plans for the coming week}
			\\I am planning to do more research on Quaternion, since we will be working on the data in terms of Quaternion rotation. I also want to finalize our problem statement and have a clear understanding of our project. Lastly, I want to start exploring the possibilities of solution that we comes up with. I tried to learn more about the correlation of the acceleration between two data and the error that the integration gives.\\

			\item \textbf{Progress since last week}
			\\My team and I work together on the problem statement this week. We also be able to get more information of this project from our clients and have a greater understanding of this project. We also have meetings that really challenge us to think more deeply about this project and make sure our team are on the same page. \\

			\item \textbf{Any problems I encountered}
			\\Although we have a better understanding than last week, it seems that we are still missing some of the points about this project. I am really grateful with the communication that our clients give to us, it is really help us to get a better understanding about this project. We also have difficulties in finding the perfect meeting time for our group. I am still not aware of my teammate’s schedule. However, we are successfully held our meeting this week and will improve on the schedule communication.\\
		\end{itemize}

		\rule{\textwidth}{0.5pt}

		\begin{center}
			\textbf{Week 4 (Oct. 17 {\textasciitilde{}} 21, 2016)}
		\end{center}
		\begin{itemize}
			\item \textbf{Plans for the coming week}
			\\I am planning to finalize our problem statement and get our final problem statement signed by our clients. I will also be working on the requirement documents and see if we have any more question about this project.\\

			\item \textbf{Progress since last week}
			\\My team and I work together on clearing the communication breakdown that happens this week between our team and our clients. I now have a clearer understanding on how to deal with a real work environment. \\

			\item \textbf{Any problems I encountered}
			\\We have a breakdown in our communication with our clients. We are trying to resolve this problem and got some tips from our teacher regarding this issues. This project going to be a learning curve for me, I have to learn more about some terminology and knowledge about hardware.\\
		\end{itemize}

		\rule{\textwidth}{0.5pt}

		\begin{center}
			\textbf{Week 5 (Oct. 24 {\textasciitilde{}} 28, 2016)}
		\end{center}
		\begin{itemize}
			\item \textbf{Plans for the coming week}
			\\I am planning to further refine our requirement documents and ask some clarification question to our clients (if any). We are also planning on getting our requirement document to be signed before the end of next week.\\

			\item \textbf{Progress since last week}
			\\We have another meetings with our clients this week on Wednesday. We clarify some stuff to move forward for our requirement documents. We also get our problem statement signed by our clients.\\

			\item \textbf{Any problems I encountered}
			\\ I have a time management problem when working on the requirement documents this week. My schedule for this week is packed with assignment and midterms. I haven't got an optimal time to do the requirement documents this week. However, I will refine our requirement documents during the weekend and hopefully can get feedback from our teacher before we send it to our clients.\\
		\end{itemize}

		\rule{\textwidth}{0.5pt}

		\begin{center}
			\textbf{Week 6 (Oct. 31 {\textasciitilde{}} Nov. 4, 2016)}
		\end{center}
		\begin{itemize}
			\item \textbf{Plans for the coming week}
			\\We will send our requirement documents to our client at the end of the week and see if we need to further refine our requirement document to be signed. I will also start to work on the Technical Review documents, looking for more options that we can do (or can't do) for this project.\\

			\item \textbf{Progress since last week}
			\\We work on the Requirement Document. Our clients has found out the ideal hardware that we will be working with for this project. Creating the requirement document makes me think more deeply about this project and how we going to achieve our goal. I got a clearer understanding on how to implements our project.\\

			\item \textbf{Any problems I encountered}
			\\The biggest problem for this week is time management. This week is a midterm week for all of us. This makes it hard for us to focus on the documents.\\
		\end{itemize}

		\rule{\textwidth}{0.5pt}

		\begin{center}
			\textbf{Week 7 (Nov. 7 {\textasciitilde{}} Nov. 11, 2016)}
		\end{center}
		\begin{itemize}
			\item \textbf{Plans for the coming week}
			\\ I will start investing my time in the Tech Review documents. I will ask more clarification question to the teacher about this documents. I will push myself in getting started with the design documents. \\

			\item \textbf{Progress since last week}
			\\We got a really good review for our requirement documents. Our clients are really pleased with the requirement documents that we send to them. I am glad that things works out and we pleased our clients with our work. \\

			\item \textbf{Any problems I encountered}
			\\The biggest problem that we faced is finding time to meet together and spend our time working together on the documents. We also have some question about our project during the weeks but our clients clarify those stuff and really help us to get the information that we need.\\
		\end{itemize}

		\rule{\textwidth}{0.5pt}

		\begin{center}
			\textbf{Week 8 (Nov. 14 {\textasciitilde{}} Nov. 18, 2016)}
		\end{center}
		\begin{itemize}
			\item \textbf{Plans for the coming week}
			\\ We are trying our best to finish our Design Documents before the thanks giving break. \\

			\item \textbf{Progress since last week}
			\\We already submitted our signed requirement documents on Monday. We have finished our Tech review documents on Wednesday noon. \\

			\item \textbf{Any problems I encountered}
			\\Working on the Tech Review documents makes me think more deeply about the project and how we actually going to build this project. This requires me to do a lot of research and make a design decision.\\
		\end{itemize}

		\rule{\textwidth}{0.5pt}

		\begin{center}
			\textbf{Week 9 (Nov. 21 {\textasciitilde{}} Nov. 25, 2016)}
		\end{center}
		\begin{itemize}
			\item \textbf{Plans for the coming week}
			\\ We will be working on the Design documents and finished it before Wednesday. We will be working on the Progress report after we submit the design documents to the clients. \\

			\item \textbf{Progress since last week}
			\\ We get the foundation for the design documents ready. We have a better idea about the progress report. \\

			\item \textbf{Any problems I encountered}
			\\It was hard to find a time to work on the document during thanks giving break.\\
		\end{itemize}

		\rule{\textwidth}{0.5pt}

		\begin{center}
			\textbf{Week 10 (Nov. 28 {\textasciitilde{}} Dec. 2, 2016)}
		\end{center}
		\begin{itemize}
			\item \textbf{Plans for the coming week}
			\\Finished up the progress report document and video. \\

			\item \textbf{Progress since last week}
			\\We get the design document submitted. \\

			\item \textbf{Any problems I encountered}
			\\Busy schedule for dead week.\\
		\end{itemize}








