\section{Introduction}
\subsection{Purpose}
The purpose of this software design document is to provide a description that sufficiently describes the system design. The system design must be specified completely to the point in which is needed for the development to proceed. The description will fully specify what is to be built, how it will be built and what expectations need to be met at completion.

\subsection{Purpose}
A HUD or a Head-Up Display provides critical information to the pilots during flight environment. Currently, the HUD obtains data from an aircraft’s mounted device called an Inertial Reference Unit (IRU), which an IRU would output precise and aligned data to the HUD through an mechanical alignment system. However, the current alignment process requires specialized equipment and epoxy which is time consuming, costly, and interrupts production line progress for the original equipment manufacturer. In addition, the resulting HUD alignment, while precise, does not compensate for airframe droop during flight. Rockwell Collins looks forward to a new alignment methodology utilizing an inexpensive microelectromechanical systems (MEMS) IRU mounted onto the HUD to infer alignment data from the aircraft’s precisely mounted and aligned IRU.

This project is to develop a feasible demonstration system as a proof of concept for Rockwell Collins, that will prove there is an algorithm that is able to output precise and aligned data with reduced installation cost utilizing the data from both the inexpensive MEMS IRU and the aircraft mounted IRU. The outcome (aligned-data) of this algorithm will compensate the alignment error correctly, and the alignment error should be within a range of one milliradian. The product will make the alignment process more dynamic and less time consuming. The dynamic alignment process also makes this new system compensate for the airframe droop that happens during the flight environment. As well as from the perspective of the industries, this product aims to improve the installation process by reducing cost and time for all parties involved.

\subsection{Overview}
The following second and third part of this document are glossary and references that we have for this document. The fourth part of this document cover the design and implementation details of this project. The design and implementation details of this document are split into six different viewpoints. These viewpoints contain context view, composition view, dependency view, interface view, interaction view, and algorithm view. 